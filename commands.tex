% Dynamically sized mid bar.
\newcommand{\bigmid}{\mathrel{\Big|}}

% ---- Colors and Notes ----
\definecolor{dblue}{RGB}{102, 120, 173}
\definecolor{dgreen}{RGB}{118, 167, 125}
\definecolor{dred}{RGB}{198, 113, 113}
\definecolor{dorange}{RGB}{230, 169, 132}
\definecolor{dtan}{RGB}{221, 215, 200}
\definecolor{dgray}{RGB}{94, 94, 94}
\definecolor{ddgray}{RGB}{46, 49, 49}

% Lights
\definecolor{dlblue}{RGB}{169, 193, 219}
\definecolor{dlgreen}{RGB}{154, 195, 157}
\definecolor{dyellow}{RGB}{246, 240, 223}


% URL
\newcommand{\durl}[1]{\textcolor{dblue}{\underline{\url{#1}}}}
\newcommand{\tx}[1]{\text{#1}}

% Circled Numbers
\newcommand*\circled[1]{\tikz[baseline=(char.base)]{\node[shape=circle,draw,inner sep=0.7pt] (char) {\footnotesize{#1}};}}
% From: http://tex.stackexchange.com/questions/7032/good-way-to-make-textcircled-numbers

% Under set numbered subset of equation
\newcommand{\numeq}[3]{\underset{\textcolor{#2}{\circled{#1}}}{\textcolor{#2}{#3}}}

\newcommand{\dnote}[1]{\textcolor{dblue}{Dave: #1}}

% ---- Abbreviations -----
\newcommand{\tc}[2]{\textcolor{#1}{#2}}
\newcommand{\ubr}[1]{\underbrace{#1}}
\newcommand{\uset}[2]{\underset{#1}{#2}}
\newcommand{\eps}{\varepsilon}
\newcommand{\KL}[2]{D_{\text{KL}}\left(#1 \mid \mid #2\right)}
\newcommand{\bKL}[2]{D_{\text{KL}}\left(#1 \bigmid \bigmid #2\right)}

% Typical limit:
\newcommand{\nlim}{\underset{n \rightarrow \infty}{\lim}}
\newcommand{\nsum}{\sum_{i = 1}^n}
\newcommand{\nprod}{\prod_{i = 1}^n}

% Add an hrule with some space
\newcommand{\spacerule}{\begin{center}\hdashrule{2cm}{1pt}{1pt}\end{center}}

% Mathcal and Mathbb
\newcommand{\mc}[1]{\mathcal{#1}}
\newcommand{\indic}{\mathbbm{1}}
\newcommand{\bE}{\mathbb{E}}

\newcommand{\longra}{\longrightarrow}
\newcommand{\longla}{\longleftarrow}
\newcommand{\ra}{\rightarrow}
\newcommand{\la}{\leftarrow}

% argmin, argmax.
\DeclareMathOperator*{\argmin}{arg\,min}
\DeclareMathOperator*{\argmax}{arg\,max}

% Quick Matrix.
\newcommand{\mat}[1]{\begin{bmatrix}#1\end{bmatrix}}

% ---- Figures, Boxes, Theorems, Etc. ----

% Basic Image
\newcommand{\img}[2]{
\begin{center}
\includegraphics[scale=#2]{#1}
\end{center}}

% Put a fancy box around things.
\newcommand{\dbox}[1]{
\begin{mdframed}[roundcorner=4pt, backgroundcolor=gray!5]
\vspace{1mm}
{#1}
\end{mdframed}
}

%  --- PROOFS ---

% Inner environment for Proofs
\newmdenv[
  topline=false,
  bottomline=false,
  rightline = false,
  leftmargin=10pt,
  rightmargin=0pt,
  innertopmargin=0pt,
  innerbottommargin=0pt
]{innerproof}

% Proof Command
%\newenvironment{dproof}{\begin{proof} \text{\vspace{2mm}} \begin{innerproof}}{\end{innerproof}\end{proof}\vspace{4mm}}
\newenvironment{dproof}[1][Proof]{\begin{proof}[#1] \text{\vspace{2mm}} \begin{innerproof}}{\end{innerproof}\end{proof}\vspace{4mm}}


% Dave Definition
\newcounter{DaveDefCounter}
\setcounter{DaveDefCounter}{1}

\newcommand{\ddef}[2]
{
\begin{mdframed}[roundcorner=1pt, backgroundcolor=white]
\vspace{1mm}
{\bf Definition \theDaveDefCounter} (#1): {\it #2}
\stepcounter{DaveDefCounter}
\end{mdframed}
}

% Block Quote
\newenvironment{dblockquote}[2]{
\begin{blockquote}
#2
\vspace{-2mm}\hspace{10mm}{#1} \\
\end{blockquote}}

% Algorithm
\newenvironment{dalg}[1]
{\begin{algorithm}\caption{#1}\begin{algorithmic}}
{\end{algorithmic}\end{algorithm}}

% Dave Table
\newenvironment{dtable}[1]
{\begin{figure}[h]
\centering
\begin{tabular}{#1}\toprule}
{\bottomrule
\end{tabular}
\end{figure}}

% For numbering the last of an align*
\newcommand\numberthis{\addtocounter{equation}{1}\tag{\theequation}}


\newtheorem{assumption}{Assumption}
\newtheorem{conjecture}{Conjecture}
\newtheorem{corollary}{Corollary}
\newtheorem{claim}{Claim}
\newtheorem{example}{Example}
\newtheorem{lemma}{Lemma}
\newtheorem{proposition}{Proposition}
\newtheorem{remark}{Remark}
\newtheorem{theorem}{Theorem}
