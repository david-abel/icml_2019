I arrived today for the best paper award talk.

\subsection{Best Paper Talk: Challenging Assumptions in Learning Disentangled Representations}

Talk by Francesco Olivier Bachem. \\

\dbox{{\bf Central Question:} Can we learn disentangled representations in a purely unsupervised way?}


{\bf Context:} Representation learning. Consider a picture of a landscape. Goal of representation learning is to learn a function $f : \mc{X} \ra \Phi$, that translates each item of interest (say, images), into a set of features that capture the important characteristics about the images. \\

Q: What might we want to have in a representation?\\

A: lots of things! One idea, though: {\it disentanglement}. \\

\ddef{Disentanglement}{A single change in a factor should lead to single change in representation.}

Example: Consider geometric shapes placed in a scene, each of a different color/shape/size. If we vary one key aspect (a square's color or size, for instance) of the original image, the representation should change in exactly one way, too. \\

{\bf Focus:} Unsupervised learning to disentangle representations (so, critically, don't get to observe the ground truth factors of variation). \\

\dbox{Contributions:
\begin{enumerate}
    \item {\it Theoretical result:} for arbitrary data, unsupervised learning of disentangled representations is impossible.
    \item {\it Large-Scale Experimental Study:} Can we learn disentangled representations without looking at the labels?
    
    $\ra$ Can reconcile and still be effective! The theorem is about {\it worst case} data. Our data likely has the right structure that we can still learn these disentangled representations.
\end{enumerate}}

{\bf Experiments:} six methods, seven data sets, six metrics, yielding 10,000 trained models and 150,000 scores. \\

$\ra$ Questions to focus on:
\begin{enumerate}[Q1]
    \item Which method should be used?
    \item How do we choose the hyperparameters?
    \item How to select the best model from a set of trained models?
\end{enumerate}

Experimental findings:
\begin{enumerate}
    \item (Toward answering Q1): Random seed and hyperparameters seem to matter more than the choice of the objective function.
    \item (Toward answering Q2): Can't identify obvious trends that can be used as a rule-of-thumb for hyperparameter selection (but, transferring across tasks seems to help).
    $\ra$ Transfer of hyperparameters to different data set and metric does not seem to substantially
    \item (Toward answering Q3):  Unsupervised model selection remains a key challenge!
\end{enumerate}

$\ra$ Implication: good runs from bad hyperparameter settings can easily outperform bad runs from good hyperparameter settings. \\

{\bf Key Takeaways:}
\begin{itemize}
    \item Role of inductive biases and supervision should be made explicit
    \item Concrete practical benefits of disentanglement should be demonstrated
    \item Sound, reproducible experimental setup with several data sets is crucial.
    \item Code: \url{github.com/google-research/disentanglement_lib/}. Also released 10,000 pre-trained models.
\end{itemize}


\spacerule


\subsection{Contributed Talks: Deep RL}

Next we have 5 minute contributed talks. They're very short so I suspect I'll have a harder time transcribing.

\subsubsection{DQN and Time Discretization~\cite{tallec2019making}}

The speaker is Corentin Tallec. \\

{\bf Point:} Time discretization actually matters in deep RL! \\

$\ra$ Usual RL algorithms with high frame rate leads to failure. Scalability often limited by algorithms; better hardware, sensors, actuators, can lead to new hyperparameters. \\

Q: Why is near continuous Q-learning failing? \\

A: As $\delta_t$ (time discretization), $Q^\pi(s,a) \ra V^\pi(s)$. As a result, at the limit, $Q$ no longer depends on actions! So we can't perform policy improvement. \\

Q: Can we solve this? \\

A: Yes! Can look at all these quantities, form new algorithms (for more, see the paper!). 

\spacerule

\subsubsection{Nonlinear Distributional Gradient TD Learning~\cite{qu2018nonlinear}}

The speaker is Chao Qu. \\
 
{\bf Background:} Distributional RL considers stochastic nature of long term return, $Z(s,a)$. \\

{\bf This Work:} Consider a distributional counterpart of Gradient TD Learning. Properties:
\begin{itemize}
    \item Converges in off policy
    \item Converges with non-linear function approximation.
\end{itemize}

$\ra$ New algorithm: Distributionl GTD2. Uses temporal distribution difference instead of the temporal difference from GTD2. 

\begin{theorem}
Under mild assumptions, Distributional GTD2 (D-GTD2) converges in the limit.
\end{theorem}

{\bf Experiments:} Show that D-GTD2 does in fact converge.

\spacerule

\subsubsection{Composing Entropic Policies using Divergence Correction~\cite{hunt2018composing}}

Talk by Jonathun Hunt. \\

Q: How do people solve complex motor control tasks such as juggling and unicycling at the same time? \\

A: Well, perhaps we tend to learn each of them independently, and then merging them together. \\

{\bf Note:} not option like! We don't do A then B, we do a new $A \circ B$ task. \\

{\bf Problem:} Given training tasks $T_1$ and $T_2$, we want to solve some merged task $T_b = T_1 + T_2$. \\

Prior work: generalized policy improvement~\cite{barreto2017successor}, and compositional optimism~\cite{Haarnoja}.

Adapt existing ideas to this transfer setting:
\begin{enumerate}
    \item Successor Features
    \item Generalized Policy Improvement
    \item Divergence correction
\end{enumerate}

$\ra$ New algorithm for composing tasks in continuous action spaces. \\

\spacerule

\subsubsection{TibGM: A Graphical Model Approach for RL~\cite{adel2019tibgm}}

The speaker is Tameem Adel. \\

{\bf Genesis:} Graphical modeling approach for 1) increasing transferability generalization, and 2) adopt precise objectives and clear interpretations. \\

$\ra$ Do this by proposing an information theoretic objective aiming at maximizing ``local" reward and a more global measure (related to transfer). \\

{\bf Contributions:}
\begin{itemize}
    \item A graphical model based on
    \item Prove a correspondence between new objective and typical reward maximization objective.
    \item An information theoretic pre-training procedure focusing on exploration.
    \item State of the art results on 16 benchmark tasks.
\end{itemize}

\spacerule

\subsubsection{Multi-Agent Adversarial IRL~\cite{yu2019multi}}

The talk is by Lantao Yu. \\

{\bf Motivation:} Performance of RL agent relies on quality of reward function. \\

$\ra$ But, it can be hard to design the right reward function! \\

One solution: learn a reward function from expert demonstrations (as in imitation learning). \\

Q: But why shoul we care about reward learning? \\

A: Well, some advantages: 1) scientific inquiry (understanding animal behavior), 2) reward function is the most succinct, robust, and transferable description of a task, 3) can be helpful if we want to reoptimize policies in new environments.  \\

$\ra$ These properties are even more desirable in multi-agent setting! \\

\ddef{Single-agent IRL}{Find a reward function that explains expert behavior}

But: intractable! Too many reward functions explain the same behavior. \\

Generalize this setting to multi-agent using Markov Games:

\ddef{Markov Game}{A multi-agent generalization of MDPs~\cite{littman1994markov}}

Solution concept to a Markov Game is a Nash Equilibrium:
\ddef{Nash Equilibrium}{When no agent can achieve higher expected reward by changing its own policy.}

Method: introduced Logistic Stochastic Best Response Equilibrium (LSBRE). Optimize the pseudolikelihood objective. \\

{\bf Experiments:} Policy imitation performance. New method achieves state of the art in both cooperative/communicative and competitive tasks. \\

{\bf Summary:} New solution concept for Markov games, gives rise to new measures of interest. Propose the first multi-agent MaxEnt IRL framework.

\spacerule

\subsubsection{Policy Consolidation for Continual RL~\cite{kaplanis2019policy}}

{\bf Motivation:} Catastrophic forgetting in neural nets! Nets tend to forget information from prior tasks. \\

$\ra$ Even happens in RL during continual learning, since distribution of seen states change over time as exploration/policy changes. \\

Agents should cope with: discrete and continuous changes to data distribution! \\

{\bf Contribiution:} Policy Consolidation agent. A bunch of agents are trained (via PPO) and connected via a KL distillation loss. \\

$\ra$ Final policy stored is the result of distilling all other policies trained. Ensures it doesn't deviate too much from prior performance. \\

{\bf Experiments:} Alternating task to explore effect of forgetting. Find their algorithm outperforms all other variants of PPO. \\

{\bf Future Work:} 1) Look at how to prioritize important memories during training, and 2) Adapt for off-policy learning. \\

\subsubsection{Off-Policy Evaluation Deep RL w/o Exploration~\cite{fujimoto2018off}}

Consider: same off-policy algorithm (DDPG) used on same dataset with two differen approaches (orange and blue). \\

Agents: 1) Orange---interacts with the environment (standard rl loop), 2) Blue---just uses data from environment but doesn't actually interact! Yet, their performance is different \\

Q: Why would these be any different? \\

A: Extrapolation error: $Q(s,a,) \la r = \gamma Q(s',a')$, where $(s,a,r,s')$ come from the dataset. Basically, you might have a bad target $Q(s',a')$.\\

\ddef{Extrapolation error}{Attempting to evaluate $\pi$ without sufficient acccess to the $(s,a)$ pairs $\pi$ visits.}

Solution: batch-constrainted RL: only choose $\pi$ so that it selects $(s,a)$ pairs that are  in the dataset and maximizes performance. \\

$\ra$ New algorithm: Batch-constrained Deep Q Learning (BCQ). \\
\begin{enumerate}
    \item Imitate dataset via generative model
    \item $\pi(s) = \argmax_a Q(s,a)$
    \item Throw in some extra magic
\end{enumerate}

Finds that BCQ greatly outperforms some existing methods (DDPG), and is quite stable. \\

\spacerule

\subsubsection{Random Expert Distillation~\cite{wang2019random}}

Talk by Ruohan Wang \\

\ddef{Imitation Learning}{Policy learning from a limited set of expert demonstrations}

Useful because: intuitive and efficient skill transfer, and can capture style/preferences of individual demonstrators. \\

$\ra$ Recent framework for IRL: Generative Adversarial Imitation Learning (GAIL). \\

But, optimization challenges; 1) training instability, and 2) sample inefficiency. \\

{\bf Main Contribution:} Random Expert Distillation (RED). Framework fir imitation learning using the estimated support of the expert policy as reward. \\

Optimize a new trajectory based loss based on a new reward function, and prove it approximates the right thing in the limit. \\

{\bf Experiment 1:} In MuJoCo, find high training stability and good sample efficiency compared to other approaches like GAIL. \\

{\bf Experiment 2:} Autonomous driving tasks with human actions as training data. Agent learns to follow the preferences of the trainer (speed, lane preference, and so on).

\spacerule

\subsubsection{Revisiting the Softmax Bellman Operator~\cite{song2019revisiting}}

The speaker is Zhao Song. \\

Recall that the Bellman Operator is a contraction. \\

$\ra$ Mellowmax operator~\cite{asadi2017alternative} is also a contraction:
\[
max_{a'} Q(s',a') \ra \sum_{a'} \frac{\exp \tau Q}{\sum_b \exp \tau Q(s',b)}
\]
Q: Is softmax really as bad as sour milk? (credit to Ron Parr). \\

{\bf Idea:} combing the max function in the target network of DDQN with softmax. \\

$\ra$ Gives rise to the SQDN, achieves higher scores than DQN on most Atari games. \\

Thus: analysis of softmax is warranted! \\

\begin{theorem}
Analysis showing: good performance, guaranteed convergence, smaller error, reduction bound, and overestimation error monotically increases w.r.t. $\tau$ (inverse temperature).
\end{theorem}

\spacerule
\subsection{Contributed Talks: RL Theory}

More RL!

\subsubsection{Distributional RL for Efficient Exploration~\cite{mavrin2019distributional}}

Talk by Hengshuai Yao. \\

{\bf Point:} Lots of sources of uncertainty in RL---estimation (in finite data regime), but also other factors like environmental stochasticity, opponents' play in game, and so on. \\

$\ra$ Exploration strategy based on uncertainty: {\it optimism under uncertainty}. In multi-armed bandits, choose arm according to:
\[
a = \argmax_k \hat{\mu}_k + c_k,
\]
where $\hat{\mu}_k$ is estimated mean payoff, and $c_k$ is some measure of uncertainty in this estimate (UCB like). \\

But! Using naive exploration bonus doesn't always work: favors actions with high intrinsic uncertainty forever. \\

{\bf New Idea:} To use optimism in face of uncertainty, perform decaying schedule of the weight of uncertainty. This ensures that as more evidence is gathered, the agent starts to exploit more.
\spacerule

\subsubsection{Optimistic Policy Optimization via Importance Sampling~\cite{papini2019optimistic}}

Talk by Matteo Papini. \\

{\bf Problem:} Policy Optimization.
\begin{itemize}
    \item Parameter space $\Theta \subseteq \mathbb{R}^d$
    \item Parapmetric policy for each $\theta \in \Theta$
    \item Each policy induces a distr. over trajectories, with $R(\tau)$ the return of the trajectory.
    \item Goal is to find he parameters that maximize the expected return:
    \[
    \max_{\theta \in \Theta} \bE[R(p_\tau \mid \theta)].
    \]
\end{itemize}

Challenge, though: exploration here is hard! \\

Q: What if we think about this as a multi-armed bandit? \\
Then:
\begin{itemize}
    \item Arms: parameters $\theta$
    \item Payoff: expected return $J(\theta)$
    \item Continuous Multi-Armed Bandit, effectively.
    \item Take advantage of arm correlation through trajectory distributions.
    \item Use Important Sampling to update return of a candidate policy.
\end{itemize}

{\bf Main result:} new algorithm ``\textsc{Optimist}", enjoys sublinear regret:
\[
Regret(T) - \tilde{O}(\sqrt{dT}).
\]

Code: \url{github.com/wolfLo/optimist} 

\spacerule

\subsubsection{Neural Logic RL~\cite{jiang2019neural}}

The talk is by Shan Luo. \\

Two main challenges of deep RL: 1) How can we generalize learned policies from one task to another?, and 2) How can we interpret these learned policies? \\

{\bf Approach:} use background knowledge to learn concepts and relations, like \texttt{grandfather(x,y)}. To do so, use {\it differentiable inductive logic programming}.

$\ra$ Idea: learn logic rules with policy gradient (a new architecture, DILP, trained with REINFORCE). \\

{\bf Experiments:} For many settings, achieve high reward compared to MLP agent (baseline). \\

\spacerule

\subsubsection{Learning to Collaborate in MDPs~\cite{radanovic2019learning}}

The talk is by Goran Radanovic. \\

{\bf Motivation:} Human-AI collaboration---consider a helper AI assisting a person in solving some task. \\

$\ra$ The agents might share a common goal, but view the task differently in some way. \\

Formal model: two-agent MDP, where agents have {\it commitments}. \\

$\ra$ Goal is to design a learning algorithm (for the first, non-human AI) that achieves sublinear regret. \\

Challenge: from the perspective of A1, the world looks like a non-stationary MDP due to the presence of A2 (the person). \\

{\bf Main Contribution:} ``Experts with Double Recency Bias", a new algorithm for this setting, based on the recency bias. 

\begin{theorem}
(Main Result): The regret of this algorithm decays as $O\left(T^{\frac{1}{4}}\right)$.
\end{theorem}

\spacerule

\subsubsection{Predictor-Corrector Policy Optimization~\cite{cheng2018predictor}}

Talk by Ching-An Cheng. \\

{\bf Problem:} Episodic policy optimization. So, agent is trying to optimize some policy $\pi(a \mid s)$, that achieves high return. \\

$\ra$ One goal: sample efficiency. We should spend time on planning/thinking before any real interactions. \\

Q: Why should we use models? \\

A: Can summarize past experience, can be more sample-efficient, and can optimize a policy efficiently without real world interactions. \\

Q: Why don't use models? \\

A: Well, models are always inexact! Weaknesses of model can be exploited in policy optimization. \\

Q: Can we reconcile these two camps? \\

A: Sure! This paper: PicColo (see talk title). A meta algorithm based on the idea that: ``should not fully trust a model but leverage only the correct part" \\

$\ra$ How can this be achieved?
\begin{itemize}
    \item Frame policy optimization as predictable online learning (POLL
    \item Design a reduction based algorithm for POL to reuse known algorithms.
    \item When translated back this gives a meta-algorithm for policy optimization.
\end{itemize}

Online learning: consider a learner and an opponent, with the learning chooses a decision $\pi_n \in \Pi$ every so often. Opponent chooses a loss function to minimize performance, rinse and repeat. \\

$\ra$ Common performance measure is {\it regret}. {\bf Idea:} Define policy optimization as an online learning process. \\

$\ra$ Algorithmically: can try typical no-regret algorithms (mirror descent), but not optimal! Want to learn faster. So, view predictability as the ability to predict future gradients. Introduce the following model:

\ddef{Predictive model}{A function that estimates the {\it gradient of future loss}
\[
\Phi_n(\pi) \approx \nabla \ell_n(\pi)
\]}

Now, develop an algorithm based on this predictability model. \\

$\ra$ Want a reduction from predictability to online learning. This is PicColo! \\

Suppose we have a predictable learning problem, idea is to turn it into an adversarial one. So:
\[
\ell_n(\cdot) = \hat{\ell}_n(\cdot) + \Delta_n(\cdot),
\]
which is a combination of prediction loss (the first term) and error (the second loss). \dnote{I didn't catch what the second source of error is}

PicColO: two steps---1) Prediction step ($\pi_n = \hat{\pi}_n - \eta_{n}\hat{g}_n$, and 2) correction step---$\hat{\pi}_{n+1} = \eta_n(g_n - \hat{g}_n$, with $g$ the return.\\

{\bf Experiments:} Compare PiColo with a variety of other algorithms on MuJoCo tasks. \\

{\bf Summary:}
\begin{itemize}
    \item PicColO can change a model-free algorithm to be faster but without the bias.
    \item Predictive model can be viewed as a unified interface for injecting prior
    \item As PicCoLO is degiend for general predictable online learning, we expect applications to other problems and domains.
\end{itemize}


\spacerule

\subsubsection{Learning a Prior over Intent via Meta IRL~\cite{xu2018learning}}

Talk by Kelvin Xu. \\

{\bf Motivation:} we often assume we have a well specified reward function! \\

Q: How can an agent infer rewards from one or a few demonstrations? \\

$\ra$ This work! Use demonstrations from previous tasks to induce tasks priors that can be used in new tasks. \\

{\bf Main idea:} use prior task information to accelerate inverse RL. \\

$\ra$ Builds on MAML (see meta-learning tutorial!). \\

{\bf Experiment:} Sprite world environment, and 2) First person navigation task. In both cases, learn task priors from meta-training task, use to learn quickly in test tasks. \\

$\ra$ Results: performs very well even only given a small number of demonstrations. \\

\spacerule

\subsubsection{DeepMDP: Learning Late Space Models for RL~\cite{gelada2019deepmdp}}

Talk by Carles Gelada. \\

{\bf Goal:} Find simple representations for RL. \\

Approach: learn a latent space model $\bar{M} = \langle \bar{\mc{S}}, \bar{\mc{A}}, \bar{R}, \bar{T} \rangle$. Based on two losses;
\begin{align}
    \bar{R}(s,a) &\approx R(s,a), \forall_{s,a} \\
    \bar{T}(s' \mid s,a) &\approx T(s' \mid s,a), \forall_{s,a,s'}.
\end{align}

Using these losses, ensure that $\phi$ is a good representation in the sense that it only throws out bad/useless policies. \\

{\bf Experiments 1:} Donut world. Image of a circle. Embedding function ends up finding a similar representation. \\

{\bf Experiments 2:} Atari w/ C51, find improvement over baselines.

\spacerule

\subsubsection{Importance Sampling Policy Evaluation~\cite{hanna2018importance}}

Talk by Josiah Hanna. \\

Note: lots of recent empirical RL success! \\

But: for RL to be successful, we must ask: ``How can RL agents get the most from small amounts of experience?". \\

{\bf Core Contribution:} Study important sampling for the RL sub-problem of policy evaluation. More specifically: replace the denominator of importance sampling with an estimate, and prove that this is justified empirically and theoretically. \\

Typical importance sampling in RL:
\[
OIS(\pi, D) = \frac{1}{m} \nsum \Pi_{t=0}^L \frac{\pi(a_t \mid s_t)}{\pi_D(a_t \mid s_t} \sum_{t=0}^L \gamma^t R_t
\]
They replace the denominator with an MLE {\it estimate} of the behavioral policy's performance:
\[
New-IS(\pi, D) = \frac{1}{m} \nsum \Pi_{t=0}^L \frac{\pi(a_t \mid s_t)}{\hat{\pi}_D(a_t \mid s_t} \sum_{t=0}^L \gamma^t R_t
\]
Paper provides theory and experiments showing this is a good thing to do. 

\spacerule

\subsubsection{Learning from a Learner~\cite{jacq2019learning}}

Talk by Alexis Jacq. \\

{\bf Goal:} Want to learn an optimal behaviour by watching others learning. \\

Assume: learning is optimizing a regularized objective:
\[
J(\pi) = \bE_\pi \left[ \sum_{t} \gamma^t(r(s_t,a_t) + \alpha H(\pi(\cdot \mid s_t))\right].
\]
The value of a state action couple is given by the fixed point of the regularized bellman equation. Moreover, the softmax is an improvement of the policy (see \citet{haarnoja2018soft}). \\

{\bf Experiments:} on MujoCo, find improvement.

\spacerule

\subsubsection{Separating Value Functions Across Time-Scales~\cite{romoff2019separating}}

Talk by Joshua Romoff. \\

Multi-step returns:
\[
G_t^k = \sum_{i=0}^{k-1} \gamma^i r_{t+1} + \gamma^k V(s_{t+k}).
\]
Can choose $k$ to trade-off between bias-variance. Or, the $\lambda$ returns:
\[
G_t^\lambda := (1-\lambda) \sum_{k=1}^\infty \lambda^{k-1} G_t^k.
\]
Many tasks are {\it discounted:} when $\gamma \ra 1$, training $V_\gamma$ is difficult. \\

Q: Why use discount factor?\\

A: Well, it makes the problem simpler. Only ever thinking about the next few rewards, so agents can learn to make decisions more easily.\\

$\ra$ But, add a lot of bias! Sometimes we don't really care about early rewards more. \\

{\bf Take a step back:} what would the ideal RL agent do? It would learn to do well quickly! \\

Ideally: want to 1) learn with a discount factor that is large but, 2) built on top of methods that are computationally and sample efficient. \\

{\bf Solution:} define a sequence of $\lambda$s:
\[
\Delta ;= \{\gamma_0, \gamma_1, \ldots, \gamma_Z\},
\]
where $\gamma_i \leq \gamma_{i+1}, \forall_i$. \\

Then, form a sequence of Bellman Equations using this sequence of $\gamma$s:
\begin{align}
W_0 &: r_t + \gamma_0 W_0(s_{t+1}) \\
W_{i > 0} &: (\gamma_i - \gamma_{i-1})V_{\gamma_{i-1}}(s_{t+1}) + \gamma_i W_i(s_{t+1}).
\end{align}
Algorithm: TD$(\Delta)$, equivalent to standard $TD(\lambda)$. \\

Analysis: Under some conditions, equivalent to standard TD$(\lambda)$.
    
    $\ra$ So, why would you do this new, sequenced thing?
    
    $\ra$ Well, equivalent when using linear function approximation. And: 1) Same learning rates for each $W$, and 2) Same $k$-step, $\gamma$, and $\lambda$ for each $W$.
    
    $\ra$ (But of course, don't have to set these parameters the same way). \\
    
Can extend TD($\Delta$) to Deep RL: 1) Share network multiple outputs, 2) Train $W$s as described, 3) Use the same of $W$s instead of $V$s \\

{\bf Experiments:} Tested in Atari, contrast TD($\Delta)$ with PPO and PPO+ (PPO but with additional parameters). \\

$\ra$ Further explore what $TD(\Delta)$ learns by plotting the value function. \\

Things they didn't try (but we should!):
\begin{itemize}
    \item Adding {\it more} $W$s.
    \item Q-learning extension (everything they tried was on policy).
    \item Model architecture for the value functions: could explore using fewer parameters, and so on.
    \item Distributed variant.
\end{itemize}

{\bf Summary:}
\begin{enumerate}
    \item Decompose the value function based off of $\gamma$
    \item Train with Bellman-like equations
    \item Improves sample efficiency
\end{enumerate}

\spacerule
\subsubsection{Learning Action Representations in RL~\cite{chandak2019learning}}

Talk by Yash Chandak. \\

{\bf Problem:} consider RL problems with {\it thousands} of possible actions! (applications might be a medical treatment, advertisement, or portfolio management). \\

$\ra$ Might capture this by learning options~\cite{sutton1999between}, but usually leads to learning {\it too many options}. \\

Q: How can we learn good action representations, but not learn too many? \\

{\bf Key Insights:}
\begin{enumerate}
    \item Actions are not independent discrete quantities
    \item There is a low dimensional structure underlying their behavior.
    \item This structure can be learned independent of the reward.
    
    \item Thus, instead of raw actions, agent can act in this new space of bejavior, behavior can be generalized to similar actions. 
\end{enumerate}

New Algorithm: 1) Supervised learning of action representations, and 2) Learn internal policy of these representations using policy gradient. \\

{\bf Experiments:} Applied algorithm to a variety of domains with rich action spaces (recommendation in photoshop, for instance), find consistently good performance. \\

\spacerule

\subsubsection{Bayesian Counterfactual Risk Minimization~\cite{london2018bayesian}}

Talk by Ben London. \\

{\bf Problem:} Learning from Logged Data to recommend music. \\

Challenges:
\begin{enumerate}
    \item Feedback is bandit feedback, so only learn from chosen actions
    \item Data is biased based on chosen logging policy.
\end{enumerate}

$\ra$ One way to counteract this bias is to use inverse propensity scoring:
\[
\argmin_{\pi} \frac{1}{n} \nsum -r_i \frac{\pi(a_i \mid x_i)}{p_i}.
\]
Can also use {\it variance regularization}, giving rise to the Counterfactual Risk Minimization (CRM) principle;
\[
\argmin_{\pi} \frac{1}{n} \nsum -r_i \frac{\pi(a_i \mid x_i)}{p_i} + \lambda \sqrt{Var(\pi, S)}
\]
$\ra$ Motivated by PAC analysis, ensures low generalization error. \\

{\bf This Work:} Bayesian view of CRM (motivated by PAC-Bayes analysis). Yields a nice generalization error bound. \\

$\ra$ Application to mixed logits.

\spacerule

\subsubsection{Per-Decision Option Counting~\cite{harutyunyan2019per}}

Talk by Anna Haratunyan. \\

{\bf Motivation:} Agents that reason over long temporal horizons. But: {\it horizon depends on choice of $\gamma$}. \\

Q: Perhaps we can do this with options~\cite{sutton1999between}? \\

A: But, doesn't really handle different horizons. \\

\dbox{{\bf Main contribution:} generalize the options framework to allow for per-option discounting.}

$\ra$ That is, add a per-decision option discount, $(\gamma_r, \gamma_p)$. \\

In this framework, they identify a bias-variance trade-off captured by these new discounts.

\spacerule

\subsubsection{Problem Dependent Regret Bounds in RL~\cite{zanette2019tighter}}

Talk by Andrea Zanette. \\

{\bf Focus:} Exploration in episodic, tabular RL. \\

$\ra$ State of the art regret bounds:
\begin{itemize}
    \item No intelligent exploration $\tilde{O}(T)$
    \item Efficient Expliration; $\tilde{O}(H\sqrt{SAT})$
    \item Lower Bound: $\tilde{O}(\sqrt{HSAT})$
    \item Below the Lower Bound: problem dependent!
\end{itemize}

Main result (requires variance of value, and scaling of reward)
\begin{theorem}
With high probability, can achieve regret of:
\[
\tilde{O}(\sqrt{\mathbb{Q}^*SAT}),
\]
where $\mathbb{Q}^*$ is a problem dependent term.
\end{theorem}

Answers an open conjecture from Jiang and Agarwal 2018 (COLT): any algorithm must suffer $H$ regret in goal-based MDPs. \\

$\ra$ Also explore effect of stochasticity on regret. \\

\spacerule

\subsubsection{A Theory of Regularized MDPs~\cite{geist2019theory}}

Talk by Matthieu Geist. \\

{\bf Motivation:} Many deep RL algorithms make use of regularization, but no general theory of how to regularize in RL. \\

$\ra$ This work: generalizes regularization in RL in two ways:
\begin{enumerate}
    \item Larger class of regularizers
    \item General modified policy iteration scheme.
\end{enumerate}

LEt $\Omega : \Delta_A \ra \mathbb{R}$ be a strongly convex function. The convex conjugate is a smoothed maximum:
\[
\forall_{q_s} \in \mathbb{R}^A, \Omega^*(q_s) = \max_{\pi_s \in \Delta_A} (\pi_s, q_s) - \Omega(\pi_s).
\]

Thus, we can regularize the Bellman Equation:
\begin{equation}
    T_{\pi,\Omega}[V] = T_\pi[V] - \Omega(V).
\end{equation}

The regularized Bellman Operators satisfy the same properties as the original ones: 1) $T_{\pi, \Omega}$ is affine, 2) Monotonicity, distributitivity, and $\gamma$-contraction. \\

$\ra$ Introduce a new algorithmic scheme, regularized policy improvement, and prove the following result:

\begin{theorem}
After $k$ iterations of reg-MPI, the loss is bounded.
\end{theorem}

$\ra$ Can also use this theory to characterize existing approaches to regularization in RL (like TRPO, DPP, and so on). \\

{\bf Summary:}
\begin{itemize}
    \item Bridges some gaps between dynamic programming and optimization
    \item Introduced temporal consistency equations, as with entropy
    \item Can generalized existing approaches to regularization, such as regularized policy gradient.
\end{itemize}


\spacerule

\subsubsection{Discovering Options for Exploration by Minimizing Cover Time~\cite{jinnai2019discovering}}

\dnote{Now, our paper!} The talk is by Yuu Jinnai. \\

{\bf Goal:} Choose options that are effective for exploration. \\

Contribution:
\begin{enumerate}
    \item Introduce an objective function for exploration called {\it cover time} (see below).
    \item Algorithm for option discovery that minimizes an upper bound on the cover time.
    
    $\ra$ Computes the {\it Fiedler} vector based on the graph induced by the graph of the MDP, uses it to minimize the cover time.
\end{enumerate}

\ddef{Cover Time}{The expected number of steps needed to visit every state.}

\begin{theorem}
The upper bound on the cover time is improved;
\[
\bE[C(G')] \leq \frac{n^2 \ln n}{\lambda_2 C(G')}
\]
\end{theorem}

Performed empirical comparison contrasting learning performance with different kinds of options. \\

\spacerule

\subsubsection{Policy Certificates: Towards Accountable RL~\cite{dann2018policy}}

Talk by Cristoph Dann. \\

Key contribution: new algorithm for episodic tabular MDPs with:
\begin{theorem}
PAC Bound;
\[
\tilde{O}\left(\frac{SAH^2}{\eps^2}\right),
\]
and a matching regret bound.
\end{theorem}

Motivation, though: {\it accountability}, not necessarily sample efficiency. \\

Q: How good will my treatment be? Is it the best possible? \\
$\ra$ What kinds of methods can answer these kinds of questions?  \\

{\bf Main Idea:} Introduce {\it policy certificates} to add accountability.
\ddef{Policy Certificate}{Confidence interval around optimal and algorithm's performance}

$\ra$ Natural extension of model-based optimistic algorithms. \\

Challenge: $Q^\pi$ is random, so it can be hard to compute confidence intervals for it. But! From optimism, we know $Q^\pi \ra Q^*$ at a known rate. So, we can bound this quantity. \\

Two main benefits:
\begin{enumerate}
    \item More accountable algorithms through accurate policy certificates.
    \item Better exploration bonuses yield minimiax-optimal PAC and regret bounds.
\end{enumerate}

\spacerule

\subsubsection{Action Robust RL~\cite{tessler2019action}}


Talk by Chen Tessler. \\

{\bf Goal:} Achieve robustness in RL/MDPs. Consider cases of abrupt disruption, highly stochastic problems, and so on. \\

$\ra$ Thus, study robust MDPs where;
\[
\pi_\alpha(\pi, \pi^') = \begin{cases}
\pi& w.p. 1-\alpha \\
\pi^'& w.p. \alpha
\end{cases}
\]

Introduce an algorithm as a a player in a two player, adversarial game. Guaranteed to converge to nash. \\

$\ra$ Propose a Deep RL variant based on this robust MDP algorithm, experiment on MuJoCo. 

\spacerule

\subsubsection{The Value Function Polytope~\cite{dadashi2019value}}

Talk by Robert Dadashi. \\

{\bf Central Question:} Can we characterize the geometry of the space of possible value functions for a given MDP? \\

Why ask this?
\begin{itemize}
    \item Relationship between policy space and value function space
    \item Better understand the dynamics of existing algorithms
    \item New formalism of representation learning in RL.
\end{itemize}

Main Result: what is the geometry of the space of value functions of a given MDP?
\begin{theorem}
The ensemble of value functions are (possibly non-convex) value polytopes.
\end{theorem}

Building blocks: 1) the line theorem (value functions of mixtures of similar policies describe a line in value function space), and 2) boundary theorem (\dnote{missed explanation}).\\

Ongoing future work: new representation learning schemes for RL, new actor-critic algorithms. \\

\dnote{And that's a wrap for Tuesday!}